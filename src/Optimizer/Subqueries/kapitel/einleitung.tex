%
% Einleitung
%
%

\chapter{Einleitung}\label{chp:Einleitung}
\pagenumbering{arabic}

\section{Problembeschreibung}\label{sct:Problembeschreibung}
SQL ist unter anderem eine m�chtige Abfragesprache, mit deren Hilfe sich viele Fragestellungen, die an relationale Daten gestellt werden k�nnen, formulieren lassen. Da SQL nach dem Vorbild der englischen Sprache entworfen wurde, erleichtert die Syntax die formale Darstellung in Umgangssprache vorliegender Informationsanforderungen. Einer der gro�en Vorteile von SQL ist die M�glichkeit, Abfragen geschachtelt zu formulieren. So l�sst sich die Fragestellung \enquote{Gib mir alle Lieferanten, deren Umsatz dem maximalen Umsatz aller Lieferanten entspricht} unter der Annahme, dass es eine Relation \enquote{Lieferanten} mit dem Schema (Name, Umsatz) gibt, formulieren als 

\sql{select * from lieferanten where umsatz = (select max(umsatz) from lieferanten)}. 

Die in einem Nebensatz formulierte Bedingung \enquote{deren Umsatz dem maximalen Umsatz aller Lieferanten entspricht} kann hier direkt als Abfrage \sql{select max(umsatz) from lieferanten} formuliert werden. Ohne eine Schachtelung lie�e sich die Fragestellung nur mit Hilfe von zwei getrennten Abfragen formulieren.

Das erweiterbare Datenbanksystem \textsc{Secondo}\footnote{ausf�hrlichere Beschreibung siehe Abschnitt \ref{sct:Beschreibung SECONDO}} stellt bereits alle Operationen zur Verf�gung, um beliebig tief geschachtelte Ausdr�cke ausf�hren zu k�nnen. Der zugeh�rige Optimierer, der f�r die �bersetzung von SQL in einen effizienten \textsc{Secondo}-Ausdruck zust�ndig ist, kann aber noch keine geschachtelten Abfragen verarbeiten. Damit ist die Formulierung komplexer geschachtelter Abfragen nur in der \textsc{Secondo}-eigenen Syntax m�glich. 

Ziel dieser Arbeit ist es, den \textsc{Secondo}-Optimierer in die Lage zu versetzen, geschachtelte Abfragen �bersetzen zu k�nnen. Nach M�glichkeit soll der Optimierer zwischen verschiedenen Ausf�hrungsstrategien die effizienteste Ausf�hrungsstrategie f�r die jeweilige �bersetzung ausw�hlen. Um die Implementierung zu testen und die Vorteile unterschiedlicher Ausf�hrungsstrategien zu demonstrieren, sollen in einer Gegen�berstellung geschachtelte Abfragen aus dem TPC-D Benchmark mit verschiedenen Ausf�hrungsstrategien ausgef�hrt und verglichen werden.

\section{Aufbau der Arbeit}\label{sct:Aufbau der Arbeit}
Die Arbeit besteht aus vier Teilen.

Im ersten Teil (Kapitel \ref{chp:Review}, ab Seite \pageref{chp:Review}) werden die technologischen und theoretischen Grundlagen dargelegt, mit denen sich geschachtelte Abfragen optimieren lassen. 

Der zweite Teil widmet sich der Problemanalyse und der ausf�hrlichen Darstellung der ausgew�hlten Algorithmen und wird in Kapitel \ref{chp:Entwurf} ab Seite \pageref{chp:Entwurf} dargestellt. Es werden die Gr�nde f�r die Auswahl der Algorithmen dargelegt und die Bedingungen f�r ihren Einsatz beschrieben.

Die Implementierung und technische Umsetzung der Algorithmen wird im dritten Teil, Kapitel \ref{chp:Implementierung} ab Seite \pageref{chp:Implementierung} geschildert. Anhand von Beispielen wird gezeigt, wie die einzelnen Schritte der Optimierung und �bersetzung geschachtelter Abfragen ablaufen.

Im vierten Teil, Kapitel \ref{chp:Leistungsbewertung} ab Seite \pageref{chp:Leistungsbewertung} werden zwei m�gliche Ausf�hrungsstrategien f�r die �bersetzung geschachtelter Abfragen anhand von geschachtelten Abfragen aus dem TPC-D Benchmark quantitativ und strukturell verglichen. 

Ab Seite \pageref{sct:Fazit} findet sich ein �berblick �ber die erreichten Ergebnisse mit einem Ausblick auf m�gliche zuk�nftige Erweiterungen, die sich aus den in dieser Arbeit nicht gel�sten Fragestellungen ergeben.

Im Anhang ab Seite \pageref{Anhang} findet sich die vom Optimierer verstandene SQL-Syntax, sowie die Implementierung der Erweiterungen des Optimierers und die Definition der implementierten Operatoren.

%
% EOF
%
%